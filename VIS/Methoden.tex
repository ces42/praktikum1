\section{Theory}

The viscosity $\eta$ can be defined via the following experimental setup: We have two infinitely large plates, the lower one is stationary and the upper one is moving with speed $v$.

The viscosity can be defined using the following formula
\[
    \eta = \frac{F b}{A v} \label{eq:vis}
\]
Where $b$ is the distance between the two plates and $F/A$ is the force per Area resulting from "friction".

In case of a Newtonian Liquid (viscosity is independent of speed) and laminar flow  we can derive the \emph{Hagen-Poiseuille-Law} 
\[
    i = \frac{\pi r^4 \Delta p}{8 \eta \ell} \label{eq:hp}
\]
which describes the flow $i$ through a cylindrical pipe of radius $r$ and length $l$, when there is a pressure difference $\Delta p$ along it.

Another Phenomenon involving viscosity is \emph{Stokes Drag}. It appear
