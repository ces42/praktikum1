\section{Theory}

The viscosity $\eta$ can be defined via the following experimental setup: We have two infinitely large plates, the lower one is stationary and the upper one is moving with speed $v$.

The viscosity can be defined using the following formula
\[
    \eta = \frac{F b}{A v} \label{eq:vis}
\]
Where $b$ is the distance between the two plates and $F/A$ is the force per Area resulting from "friction".

In case of a Newtonian liquid (viscosity is independent of speed) and laminar flow  we can derive the \emph{Hagen-Poiseuille-Law} 
\[
    i = \frac{\pi r^4 \Delta p}{8 \eta \ell} \label{eq:hp}
\]
which describes the flow $i$ through a cylindrical pipe of radius $r$ and length $l$, when there is a pressure difference $\Delta p$ along it.

Another Phenomenon involving viscosity is \emph{Stokes' drag}. This phenomenon occurs when a ball of radius $r$ falls with speed $v$ through a liquid flowing around it in laminar fashion. The drag on the ball is given by
\[
    F = 6 \pi \eta r v \label{eq:stokes}
\]
If we assume that the ball is falling with terminal velocity, i.\,e. there is an equilibrium of the drag, the buoyancy and the gravitational Force exerted on the ball, we get
\[
    \eta = \frac{2 r^2 g}{9 v}(\rho_\mathrm{ball} - \rho_\mathrm{fl}) \label{eq:termid}
\]
However this equation only holds in the Limit $r \to 0$. 
If we assume that the ball is falling through a cylinder of radius $R$ and not near the caps we get the corrected formula
\[
    \eta = \frac{2 r^2 g}{9 v \left( 1 + 2.4 \frac{r}{R} \right)}(\rho_\mathrm{ball} - \rho_\mathrm{fl}) \label{eq:term}
\]




