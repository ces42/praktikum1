\subsection{Ubbelohde Viscometer} \label{sec:aufgabe1}
% !TeX spellcheck = en_US
Our measurements yield a mean of $\unit[339.2]{s}$ with a corrected standard deviation of the mean of $\unit[0.4]{s}$. It is plausible to assume that there could be a systematic error of about $\unit[0.3]{s}$ (due to the fact that everybody tends to stop the time a bit too late at the end, for example). The resulting total error is $\unit[0.5]{s}$. The kinematic viscosity is calculated as
\[
    \nu = K t = \unit[(957.6 \pm 1.4)\e{-9}]{m^2/s}
\]
Where K is given as $K = \unit[0.002832\e{-6}]{m^2/s^2}$. The manufacturer does not provide an Error for this constant, but since it is provided with 4 significant figures, we assume that the error is smaller than $0.5\%$ which should make it negligible compared to the larger (and independet) error of $t$.

The temperature of the water was measured as $(22.8 \pm 0.5)\C$, which according to the Literature%
\footnote{\url{http://www.wolframalpha.com/input/?i=density+of+water+at+22.8+celsius}. The values from Wolfram Alpha perfectly agree with those provided by \\ \textsc{David R. Lide}: Handbook of Chemistry and Physics, CRC Press, 85. Edition, Page 6--2}
corresponds to a density of $\rho = \unit[(997.6 \pm 0.1)]{kg\,m^{-3}}$. We therefore calculate for the viscosity
\[
    \eta =  \nu \rho = \unit[(0.9553 \pm0.0014)]{mPa\,s}
\]
This value does not agree very well with the value found in Literature%
\footnote{Source: Mathematica}
of $\eta = \unit[(0.937 \pm 0.011)]{mPa\, s}$. This can have multiple reasons, most importantly (we assume) the difference of earth acceleration at the place of creation of the Ubbelohde viscometer and Garching, which is also included in the constant $K$, or maybe even pollution.\\
But nevertheless, the value is quite near the values found in literature. Comparing them with the values from section \ref{sec:aufgabe3} they lie within error range. The errors from section \ref{sec:aufgabe3} are quite big, but unlike the viscosity from this section, they lie perfectly between the values from literature and are therefore reasonable.
