\section{Verwendete Methoden}
\subsection{Wellen}
Man unterscheidet zwei Arten von Wellen. Longitudinalwellen (Schwingungsrichtung parallel zur Ausbreitungsrichtung) und Transversalwellen (Schwingungsrichtung senkrecht zur Ausbreitungsrichtung). Longitudinalwellen können sich in Gasen, Flüssigkeiten und Festkörpern ausbreiten, Transversalwellen nur in Festkörpern. Betrachtet man ein Masseteilchen am Ort $x$ zum Zeitpunkt $t$, ergibt sich für die Auslenkung $a$ folgender Zusammenhang:

\begin{equation}
a(x, t) = A \cdot \cos (\omega \cdot t - k \cdot x + \varphi)
\end{equation}
Wobei $T$ die Schwingungsdauer, $A$ die Amplitude, $\lambda$ die Wellenlänge $\omega = 2\pi /T$ die Kreisfrequenz, $k = 2\pi /\lambda$ und $\varphi$ die Phase ist.\\
Hieraus lässt sich die Ausbreitungsgeschwindigkeit berechnen mit:
\begin{equation}
v = \frac{\lambda}{T} = \frac{\omega}{k} = f \cdot \lambda
\end{equation}
Für die Frequenz $f = 1/T$.\\
Für die Geschwindigkeit von Longitudinalwellen in langen Stäben gilt für das Elastizitätsmodul $E$ und die Dichte $\rho$ die Näherung:\footnote{Ursprung der Formel: Aufgabenstellung}
\begin{equation}
\sqrt{\frac{E}{\rho}}
\end{equation}