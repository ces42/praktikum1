\subsection{Schallgeschwindigkeit in Feststoffen}
Tabelle~\ref{tb:res} Zeigt die Messwerte und Schallgeschwindigkeiten $c_\mathrm{S}$ für die verwendeten Materiale. Der Fehler der Länge liegt bei unter $\unit[3]{mm}$ und trägt praktisch nichts zum Fehler der Schallgeschwindigkeit bei. 
Der Fehler der Laufzeit wurde als die Standardabweichung der gemessenen Laufzeiten (jeweils 9 Datenpunkte) berechnet. Den systematischen Fehler des Oszilloskops braucht man nicht zu berücksichtigen, weil er viel kleiner als die Standardabweichung.

\begin{table}
    \centering
\begin{tabular}[h]{c c c c c }
    Material & Länge in $\unit{cm}$ & Laufzeit in $\unit{\mu s}$ & $c_\mathrm{S}$ in $\unit{m/s}$ & Elastizitätsmodule in $\unit{GPa}$\\\hline
    Cu & $1.498$ & $770 \pm 40$ & $3870 \pm 200$ & $134 \pm 14$\\
    Al & $1.499$ & $610 \pm 30$ & $4950 \pm 260$ & $66 \pm 7$\\
    PVC & $1.024$ & $1310 \pm 50$ & $1560 \pm 60$ & $3.41 \pm 0.29$\\
\end{tabular}
\caption{Ergebnisse für Kupfer, Aluminium und PVC}
\label{tb:res}
\end{table}

Für Kupfer und Aluminium liefert die Literatur%
\footnote{David R. Lide: Handbook of Chemistry and Physics, CRC Press, 85. Auflage, Seite 14-41}
Werte für die Schallgeschwindigkeit von $\unit[3810]{m/s}$ und $\unit[5000]{m/s}$. Unsere Messwerte stimmen im Rahmen ihres  Fehlers mit diesen Werten überein. Für PVC haben wir nur widersprüchliche Werte, die zum Teil deutlich über und deutlich unter unserem Wert lagen gefunden.

Gleichung~\ref{eq:elast} liefert
%
\begin{equation*}
    E = \rho {c_\mathrm{S}}^2
\end{equation*}
%
womit die Werte in Tabelle~\ref{tb:res} berechnet wurden. Da der relative Fehler in $c_\mathrm{s}$ deutlich größer ist als der in $\rho$ (und zudem doppelt so stark in den Endfehler eingeht), ergibt sich der Fehler von $E$ zu
\[
    \Delta E \approx \rho c_\mathrm{s} \cdot \Delta c_\mathrm{S}
\]

