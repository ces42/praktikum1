\documentclass[12pt,a4paper]{article}
\usepackage[utf8]{inputenc}
\usepackage[german]{babel}
\usepackage[T1]{fontenc}
\usepackage{amsmath}
\usepackage{amsfonts}
\usepackage{amssymb}

\usepackage{pgf,tikz}
\usepackage{pgfplots}
\usepackage{csvsimple}
\usepackage{epstopdf}
\usepackage{units}

\usepackage{hyperref}
\usepackage{graphicx}
\usepackage[left=2cm,right=2cm,top=2cm,bottom=2cm]{geometry}

\title{Akustik (AKU)}
\author{Kilian Brenner und Carlos Esparza \\ Team 13, Gruppe 6}

\date{19.03.2018}


\pgfplotsset{compat=newest,
    tick label style={font=\small},
    label style={font=\small},
    legend style={font=\footnotesize}
}

%\tikzset{every mark/.append style={scale=0.3}}
\newlength{\plotheight}
\newlength{\plotwidth}
\newlength{\imgheight}
\setlength{\plotwidth}{\textwidth}
\setlength{\plotheight}{11cm}

\newcommand{\e}[1]{\cdot\!10^{#1}}
\renewcommand{\d}{\mathrm{d}}
\renewcommand{\phi}{\varphi}


\begin{document}
\maketitle
\tableofcontents
\newpage

\section{Einleitung}

\section{Verwendete Methoden}
\subsection{Das physikalische Pendel}
Anders als in einem mathematischen Pendel, in dem ein Massepunkt $m'$ an einem masselosen Faden der Länge $l'$ hängt, wird im physikalischen Pendel die Volumenausehnung des um eine Drehachse schwingenden Körpers berücksichtigt. Das beschleunigende Drehmoment $M$ wird durch den Winkel $\varphi$

\section{Experimentelles Vorgehen}
In der Versuchsdurchführung soll in jeweils drei Versuchsdurchläufen der Gefrierpunkt von destilliertem Wasser sowie von einer Salzlösung (in diesem Fall: $NaNO_3$) festgestellt werden. Zusätzlich soll der Temperaturverlauf des Wassers/der Lösung alle 5 Sekunden protokolliert werden. Die Temperatur soll mithilfe eines Thermistors bestimmt werden, der im Anschluss kalibriert wird. Wie in Abb. \ref{fig:aufbau} dargestellt, wird das Reagenzglas mit der Testflüssigkeit in ein Kältebad mit etwa $\unit[-6]{^\circ C}$ gegeben. Sobald sich der Transistor wie in der Abbildung skizziert in der Lösung befinden, wird mit der Messung begonnen. Während der Messung wird mithilfe der Rührer sowohl das Kältebad als auch die Testflüssigkeit durchmischt, um eine homogene Temperaturverteilung zu gewährleisten. Ist der Widerstand (also die Temperatur) für längere Zeit konstant, ist der Gefrierpunkt erreicht worden, welcher durch die anschließende Kalibrierung ermittelt werden kann.\\
Das Kältebad wird auf $\unit[-6]{^\circ C}$ gebracht, indem der Gefrierpunkt des Wassers mithilfe von Streusalz ($NaCl$) erniedrigt wird. Anschließend wird das Gemisch mithilfe von Eis auf die entsprechende Temperatur gebracht. Die Temperatur wird hierbei mit einem normalen Digitalthermometer gemessen.


\begin{figure}
\begin{center}
\includegraphics[scale=0.5]{Bilder/Versuchsaufbau.png}
\caption{Versuchsaufbau}
\label{fig:aufbau}
\end{center}
\end{figure}

\section{Ergebnisse}
\subsection{Aufgabe 3}
In Aufgabe drei sollte die Berechnung Stoffmenge über eine Interpolation zu einem Volumen erfolgen. Hierzu wird ein $\frac{1}{V}$ -- $pV$--Diagramm erstellt und der y-Achsenabschnitt über ein Ausgleichspolynom 3ter Ordnung ermittelt (Reihenentwicklung).\footnote{Berechnung durch Origin} Bei der Interpolation ergibt sich für $y$ bei $x=0$ der Wert $y=\unit[7.563]{J}  \pm 0,39$. Aus der Gleichung für ideale Gase, ergibt sich für den Druck $p [\mathrm{100kPa}]$, das Volumen $V [\mathrm{cm^3}]$, die Stoffmenge $n_2 [\mathrm{mmol}]$ (Für die kritische Temperatur von Gruppe zwei von $\unit[50]{^\circ C}$), die allgemeine Gaskonstante $R [\mathrm{\frac{J}{mol \cdot K}}]$ und die Temperatur $T [\mathrm{K}]$:

\begin{align}
p \cdot V &= n_2 \cdot R \cdot T\\
n_2 &= \frac{p \cdot V}{R \cdot T} \approx \unit[(22.86 \pm 0.02)]{mmol} \notag
\end{align}
Der Fehler ergibt sich durch:
\[
F = \sqrt{F_{stat}^2 + F_{sys}^2}
\]
Wobei $F_{stat} = \unit[0.01]{mmol}$. Der systematische Fehler des Thermometers liegt bei $\unit[0.2]{^\circ C}$, somit ergibt sich ein systematischer Fehler für $n$ von $F_{sys} = \unit[0.017]{mmol}$, der Gesamtfehler ergibt sich demnach zu $F = \unit[0.022]{mmol}$. \\
Für die später benötigten Stoffmengen $n_1$ ($\unit[47.5]{^\circ C}$) und $n_3$ ($\unit[55]{^\circ C}$) ergeben sich folgende Werte.
\begin{align*}
n_1 &= \unit[(22.86 \pm 0.02)]{mmol}\\
n_3 &= \unit[(20.36 \pm 0.02)]{mmol}
\end{align*}

\section{Diskussion und Zusammenfassung}
Mithilfe von Laufzeitmessung wurde in den ersten beiden Versuchen die Schallgeschwindigkeit von Luft, Kupfer, PVC und Aluminium berechnet. Im ersten Versuch ergeben sich sehr große Fehlerwerte, innerhalb des Fehlerbereichs stimmen die Werte allerdings mit Literaturwerten überein, anders als im zweiten Versuch, in dem die Werte einigermaßen nah an den Literaturwerten liegen, allerdings außerhalb des -- recht kleinen -- Fehlerbereichs.\\
In Versuch drei ist die Schallgeschwindigkeit erstaunlich genau bestimmt worden vor allem unter der Berücksichtigung der Tatsache, dass Umgebungsgeräusche einen sehr großen Störfaktor gebildet haben (insbesondere bei hohen Frequenzen) und die Länge, für die Resonanzen entstehen, nach Augenmaß ermittelt wurde.

\input{Zusammenfassung.tex}


% \begin{appendix}
% \addcontentsline{toc}{section}{Anhang}
% \input{Fehlerrechnung.tex}
% \end{appendix}


\end{document}















