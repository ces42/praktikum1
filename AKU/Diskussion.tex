\section{Diskussion und Zusammenfassung}
Mithilfe von Laufzeitmessung wurde in den ersten beiden Versuchen die Schallgeschwindigkeit von Luft, Kupfer, PVC und Aluminium berechnet. Im ersten Versuch ergeben sich sehr große Fehlerwerte, innerhalb des Fehlerbereichs stimmen die Werte allerdings mit Literaturwerten überein, anders als im zweiten Versuch, in dem die Werte einigermaßen nah an den Literaturwerten liegen, allerdings außerhalb des -- recht kleinen -- Fehlerbereichs.\\
In Versuch drei ist die Schallgeschwindigkeit erstaunlich genau bestimmt worden vor allem unter der Berücksichtigung der Tatsache, dass Umgebungsgeräusche einen sehr großen Störfaktor gebildet haben (insbesondere bei hohen Frequenzen) und die Länge, für die Resonanzen entstehen, nach Augenmaß ermittelt wurde.