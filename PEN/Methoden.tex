\section{Verwendete Methoden}
\subsection{Das physikalische Pendel}
Anders als in einem mathematischen Pendel, in dem ein Massenpunkt $m'$ an einem masselosen Faden der Länge $l'$ hängt, wird im physikalischen Pendel die Volumenausdehnung des um eine Drehachse schwingenden Körpers berücksichtigt. Das beschleunigende Drehmoment $M$ wird durch den Winkel $\varphi$ und den Abstand $l_\n{1}$ zum Massenmittelpunkt folgendermaßen ausgedrückt:

\begin{equation*}
M = l \cdot m \cdot g \cdot \sin \varphi
\end{equation*}

Zusammen mit dem Trägheitsmoment $J$ ergibt sich folgende Gleichung

\begin{equation}
J \cdot \ddot{\varphi} = - m \cdot g \cdot l_1 \cdot \sin \varphi
\label{eq:basis}
\end{equation}

Es ergibt sich also eine harmonische Schwingung mit Schwingungsdauer (unter Benutzung der Kleinwinkelnäherung $\sin \varphi \approx \varphi$)

\begin{equation}
T = 2\pi \sqrt{\frac{J}{mgl_1}} = 2\pi \cdot  \sqrt{\frac{1}{g}\left(\frac{J_\n{S}}{ml_1}+l_1\right)}
\label{eq:period}
\end{equation}

Wobei die letzte Umformung mithilfe des Satzes von Steiner umgeformt wurde und $J_S$ das Trägheitsmoment durch den Schwerpunkt beschreibt.

\subsection{Das Reversionspendel}
Möchte man Formel \ref{eq:period} nach $l_1$ umstellen, stellt man fest, dass zwei Lösungen existieren müssen, d.h. es existieren zwei Längen als Abstand vom Schwerpunkt, welche die gleiche Schwingungsdauer haben. Die zweite Länge bezeichnen wir im folgenden als $l_2$. Liegen die beiden Achsen und der Schwerpunkt auf einer geraden, wird dies ein Reversionspendel genannt.\\
Das mathematische Pendel und das physikalische Pendel haben die selbe Schwingungsdauer, wenn 

\begin{equation*}
l_\n{r} = \frac{J_\n{S}+ml^2_1}{ml_1} = l_1 + l_2
\end{equation*}

Dies ergibt sich direkt aus der Formel der Schwingungsdauer bzw. der Lösung der Gleichung \ref{eq:period} nach $l$. $l_\n{r}$ heißt reduzierte Pendellänge. Für die Schwingungsdauer ergibt sich damit folgenden Zusammenhang:

\begin{equation*}
T^2 = 4\pi^2 \cdot \frac{l_\n{r}}{g}
\end{equation*}

also dementsprechend für die Erdbeschleunigung:

\begin{equation}
g = 4\pi^2 \cdot \frac{l_\n{r}}{T^2}
\label{eq:gravity}
\end{equation}

\subsection{Gekoppelte Pendel}
Bei gekoppelten Pendeln werden zwei Pendel mit einer Feder (Federkonstante $d$) verbunden. Im folgenden Werden beide Pendel als gleich (gleiche Schwingungsdauer $T$ und gleiches Trägheitsmoment $J$) angenommen. $\Psi_1$ und $\Psi_2$ beschreiben im folgenden die Auslenkung des ersten bzw. des zweiten Pendels aus der neu entstehenden Ruhelage mit der Feder. Als Vereinfachung ist im Folgenden $D = m \cdot g \cdot l$. Das Drehmoment, welches nun auf die Pendel wirkt, ist folgendes:

\begin{align*}
M_1 &= -D \cdot \psi_1 - d \cdot r^2(\Psi_2 - \Psi_1)   \label{eq:drehmoment1} \\
M_2 &= -D \cdot \psi_2 - d \cdot r^2(\Psi_2 - \Psi_1)   \label{eq:drehmoment2}
\end{align*}

Mit den Abkürzungen $k^2 := \frac{dr^2}{J}$ sowie $w_{gl}^2 := \frac{D}{l}$ sowie den Substitutionen $X := \Psi_1 + \Psi_2$ und $Y := \Psi_1 - \Psi_2$.

\begin{align}
X(t) &= A_1 \cdot \sin (\omega_{\n{gl}}t) + A_2 \cdot \cos (\omega_{\n{gl}}t)		\label{eq:subst1} \\
Y(t) &= A_3 \cdot \sin (\omega_{\n{geg}}t) + A_4 \cdot \cos (\omega_{\n{geg}}t) 	\label{eq:subst2}
\end{align}

Für die Schwingungsfrequenz $\omega_{\n{gl}}$ bei der die Pendel in Phase schwingen, können diese beiden Therme benutzt werden, um die Schwingung darzustellen. Mithilfe von Additionstheoremen kann nach $\Psi(t)$ umgestellt werden

\begin{align}
\Psi_1 (t) &= 2A \cdot \sin (\omega_\n{M} \cdot t) \cdot \cos (\omega_\n{S} \cdot t)	\label{eq:psi1} \\
\Psi_2 (t) &= 2A \cdot \cos (\omega_\n{M} \cdot t) \cdot \sin (\omega_\n{S} \cdot t)	\label{eq:psi2}
\end{align}

wobei

\begin{align}
\omega_\n{M} &= \frac{\omega_\n{gl} + \omega_\n{geg}}{2}	\label{eq:avg} \\
\omega_\n{S} &= \frac{\omega_\n{geg} - \omega_\n{gl}}{2}	\label{eq:schwebung}
\end{align}

wobei $\omega_\n{M}$ die mittlere Kreisfrequenz und $\omega_\n{S}$ die Schwebungskreisfrequenz ist.\\
Zudem wird der Kopplungsgrad $K$ definiert:

\begin{equation}
K = \frac{\omega_\n{geg}^2 - \omega_\n{gl}^2}{\omega_\n{gl}^2 + \omega_\n{geg}^2} = \frac{d \cdot r^2}{D + d \cdot r^2}	\label{eq:kopplung}
\end{equation}






