\section{Diskussion und Zusammenfassung}
In diesem Versuch wurde zum einen die Erdbeschleunigung mit einem Reversionspendel bestimmt, sowie das Verhalten von Schwebungen von zwei mit einer Feder gekoppelten Pendeln untersucht.\\
Im Versuch zur Bestimmung der Erdgravitation in Garching/München stimmen die berechneten Werte sehr genau mit Literaturwerten (für München) überein. Auch beim zweiten Versuch stimmen die berechneten und die ermittelten Werte für die Schwebungsfrequenzen sehr genau mit den theoretischen Werten überein und passen innerhalb des Fehlerbereichs genau zueinander. Auch für die mittlere Kreisfrequenz passen die Werte zusammen, obwohl es (wie in Abbildung \ref{fig:gegen1} sehr gut zu sehen) Abweichungen bei der anfänglichen Auslenkung gibt, sodass keine reine Gleich- oder gegenphasige Schwingung bzw. Schwebung existiert.