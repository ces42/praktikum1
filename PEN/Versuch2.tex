\section{Gekoppelte Pendel}
Durch Ermittlung der Schwingungsfrequenz der Phase und der Gegenphase, kann mithilfe von Gleichung \ref{eq:schwebung} die Frequenz



\begin{table}
\resizebox{\linewidth}{!}{
% \begin{tabular}{|c|c|c|c|c|c|}
% \hline 
% • & 28.2 mm & 53.2 mm & 78.2 mm & 102.2 mm & 28.2 mm / Feder 2 \\ 
% \hline 
% $T_\mathrm{gl}/2$ & $\unit[(0.95841 \pm 1.1\e{-5})]{s}$ & $\unit[(0.95798 \pm 1.3\e{-5})]{s}$ & $\unit[(0.95724 \pm 3.2\e{-5})]{s}$ & $\unit[(0.95781 \pm 1.1\e{-5})]{s}$ & $\unit[(0.95826 \pm 1.3\e{-5})]{s}$ \\ 
% \hline 
% $T_\mathrm{geg}/2$ & $\unit[(0.92861 \pm 1.8\e{-5})]{s}$ & $\unit[(0.86739 \pm 1.1\e{-5})]{s}$ & $\unit[(0.79031 \pm 1.1\e{-5})]{s}$ & $\unit[(0.71542 \pm 1.1\e{-5})]{s}$ & $\unit[(0.91844 \pm 2.0\e{-5})]{s}$ \\ 
% \hline 
% $\omega_\n{S}$ berechnet & $\unit[(0.0526 \pm 0.0044)]{s^{-1}}$ & $\unit[(0.1712 \pm 0.0044)]{s^{-1}}$ & $\unit[(0.3466 \pm 0.0044)]{s^{-1}}$ & 
% $\unit[(0.5556 \pm 0.0044)]{s^{-1}}$ & $\unit[(0.0711 \pm 0.0044)]{s^{-1}}$ \\ 
% \hline 
% $\omega_\n{S}$ experimentell & $\unit[(0.0519 \pm 0.0044)]{s^{-1}}$ & $\unit[(0.1708 \pm 0.0044)]{s^{-1}}$ & $\unit[(0.346988 \pm 16\e{-6})]{s}$ & val7 & 
% $\unit[(0.0706\pm0.0044)]{s^{-1}}$ \\ 
% \hline 
%\end{tabular} 
\def\arraystretch{1.2}
\begin{tabular}{ccccc}
    Loch bei & $T_\mathrm{gl}/2$ & $T_\mathrm{geg}/2$ & $\omega_\n{S}$ berechnet & $\omega_\n{S}$ experimentell\\[1mm]
    \hline
    28.2 mm & $\unit[0.958410(11)]{s}$ & $\unit[0.928610(18)]{s}$ & $\unit[0.0526(31)]{s^{-1}}$ & $\unit[0.0519(31)]{s^{-1}}$\\
    \hline
    53.2 mm & $\unit[0.957980(13)]{s}$ & $\unit[0.867390(11)]{s}$ & $\unit[0.1712(31)]{s^{-1}}$ & $\unit[0.1708(31)]{s^{-1}}$\\
    \hline
    78.2 mm & $\unit[0.95724(3)]{s}$ & $\unit[0.790310(11)]{s}$ & $\unit[0.3466(31)]{s^{-1}}$ & $\unit[0.3470(31)]{s^{-1}}$\\
     \hline
    102.2 mm & $\unit[0.957810(11)]{s}$ & $\unit[0.715420(11)]{s}$ & $\unit[0.5556(31)]{s^{-1}}$ & $\unit[0.5555(31)]{s^{-1}}$\\
    \hline
    28.2 mm (Feder 2)  & $\unit[0.958260(13)]{s}$  & $\unit[0.918440(20)]{s}$  & $\unit[0.0711(31)]{s^{-1}}$  & $\unit[0.0706(31)]{s^{-1}}$\\
\end{tabular}
}
\caption{Angegeben wird der Abstand der Feder zum Aufhängepunkt und die halbe Schwingungsdauer der Gleich-/Gegenschwingungen bzw. die daraus berechneten und experimentell bestimmten einhüllenden Frequenzen $\omega_\n{S}$.}
\label{tb:values}
\end{table}

Die angegebenen Fehler beziehen sich auf die Regression. Allerdings weichen die Werte für die Schwingungen in Phase, die in der Theorie gleich sind, da die Feder nicht aus ihrer Gleichgewichtslage bewegt wird. Mithilfe dieser Werte können wir unseren systematischen Fehler auf etwa $\unit[0.001]{s}$ abschätzen. Für Kreisfrequenzen ergibt sich somit ein Fehler von ca. $\unit[0.0031]{s^{-1}}$. Die Berechnung des gesamten Fehlers berechnen wir also durch:

\begin{equation*}
\Delta \omega_\n{S} = \sqrt{
\left(\frac{1}{2} \cdot \Delta\omega_\n{gl}\right)^2
+ 
\left(\frac{1}{2} \cdot \Delta\omega_\n{geg}\right)^2
+
\left(\frac12 \unit[0.0031]{s^{-1}} + \frac12 \unit[0.0031]{s^{-1}} \right)^2
}
\end{equation*}

Wie hier sehr schön gesehen werden kann, spielen die Fehler durch die Regression kaum eine Rolle. Die größte Fehlerursache sind Störfaktoren wie z.B. unterschiedlich starke Reibung (durch unterschiedlich große Auslenkung usw.) welche sich in unterschiedlichen Messungen ändern.



