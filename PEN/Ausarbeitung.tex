\documentclass[12pt,a4paper]{article}
\usepackage[utf8]{inputenc}
\usepackage[ngerman]{babel}
\usepackage[T1]{fontenc}
\usepackage{amsmath}
\usepackage{amsfonts}
\usepackage{amssymb}

\usepackage{pgf,tikz}
\usepackage{pgfplots}
\usepackage{csvsimple}
\usepackage{epstopdf}
\usepackage{units}
\usepackage{microtype}
\usepackage{esvect}

\usepackage{hyperref}
\usepackage{graphicx}
\usepackage[left=2cm,right=2cm,top=2cm,bottom=2cm]{geometry}
\setlength{\parindent}{0pt}

\title{Pendel (PEN)}
\author{Kilian Brenner und Carlos Esparza \\ Team 13, Group 6}

\date{2.04.2018}


\pgfplotsset{compat=newest,
    tick label style={font=\small},
    label style={font=\small},
    legend style={font=\footnotesize}
}

%\tikzset{every mark/.append style={scale=0.3}}
\newlength{\plotheight}
\newlength{\plotwidth}
\newlength{\imgheight}
\setlength{\plotwidth}{\textwidth}
\setlength{\plotheight}{11cm}

\newcommand{\e}[1]{\cdot\!10^{#1}}
\renewcommand{\emph}{\textbf}
\newcommand{\C}{\unit{\, ^\circ C}}
\newcommand{\abs}[1]{\left| #1 \right|}
\newcommand{\n}[1]{\mathrm{#1}}

\usepackage{mathtools}
%\DeclarePairedDelimiter{\abs}{\lvert}{\rvert}


\begin{document}
\maketitle
\tableofcontents
\newpage

\section{Einleitung}
In diesem Versuch werden die Eigenschaften von verschiedenen Pendeln untersucht. Außerdem erhalten wir eine Methode um die lokale Erdbeschleunigung $g$ experimentell zu bestimmen. 

\section{Verwendete Methoden}
\subsection{Das physikalische Pendel}
Anders als in einem mathematischen Pendel, in dem ein Massepunkt $m'$ an einem masselosen Faden der Länge $l'$ hängt, wird im physikalischen Pendel die Volumenausehnung des um eine Drehachse schwingenden Körpers berücksichtigt. Das beschleunigende Drehmoment $M$ wird durch den Winkel $\varphi$

\section{Experimentelles Vorgehen}
In der Versuchsdurchführung soll in jeweils drei Versuchsdurchläufen der Gefrierpunkt von destilliertem Wasser sowie von einer Salzlösung (in diesem Fall: $NaNO_3$) festgestellt werden. Zusätzlich soll der Temperaturverlauf des Wassers/der Lösung alle 5 Sekunden protokolliert werden. Die Temperatur soll mithilfe eines Thermistors bestimmt werden, der im Anschluss kalibriert wird. Wie in Abb. \ref{fig:aufbau} dargestellt, wird das Reagenzglas mit der Testflüssigkeit in ein Kältebad mit etwa $\unit[-6]{^\circ C}$ gegeben. Sobald sich der Transistor wie in der Abbildung skizziert in der Lösung befinden, wird mit der Messung begonnen. Während der Messung wird mithilfe der Rührer sowohl das Kältebad als auch die Testflüssigkeit durchmischt, um eine homogene Temperaturverteilung zu gewährleisten. Ist der Widerstand (also die Temperatur) für längere Zeit konstant, ist der Gefrierpunkt erreicht worden, welcher durch die anschließende Kalibrierung ermittelt werden kann.\\
Das Kältebad wird auf $\unit[-6]{^\circ C}$ gebracht, indem der Gefrierpunkt des Wassers mithilfe von Streusalz ($NaCl$) erniedrigt wird. Anschließend wird das Gemisch mithilfe von Eis auf die entsprechende Temperatur gebracht. Die Temperatur wird hierbei mit einem normalen Digitalthermometer gemessen.


\begin{figure}
\begin{center}
\includegraphics[scale=0.5]{Bilder/Versuchsaufbau.png}
\caption{Versuchsaufbau}
\label{fig:aufbau}
\end{center}
\end{figure}

\section{Ergebnisse}
\subsection{Aufgabe 3}
In Aufgabe drei sollte die Berechnung Stoffmenge über eine Interpolation zu einem Volumen erfolgen. Hierzu wird ein $\frac{1}{V}$ -- $pV$--Diagramm erstellt und der y-Achsenabschnitt über ein Ausgleichspolynom 3ter Ordnung ermittelt (Reihenentwicklung).\footnote{Berechnung durch Origin} Bei der Interpolation ergibt sich für $y$ bei $x=0$ der Wert $y=\unit[7.563]{J}  \pm 0,39$. Aus der Gleichung für ideale Gase, ergibt sich für den Druck $p [\mathrm{100kPa}]$, das Volumen $V [\mathrm{cm^3}]$, die Stoffmenge $n_2 [\mathrm{mmol}]$ (Für die kritische Temperatur von Gruppe zwei von $\unit[50]{^\circ C}$), die allgemeine Gaskonstante $R [\mathrm{\frac{J}{mol \cdot K}}]$ und die Temperatur $T [\mathrm{K}]$:

\begin{align}
p \cdot V &= n_2 \cdot R \cdot T\\
n_2 &= \frac{p \cdot V}{R \cdot T} \approx \unit[(22.86 \pm 0.02)]{mmol} \notag
\end{align}
Der Fehler ergibt sich durch:
\[
F = \sqrt{F_{stat}^2 + F_{sys}^2}
\]
Wobei $F_{stat} = \unit[0.01]{mmol}$. Der systematische Fehler des Thermometers liegt bei $\unit[0.2]{^\circ C}$, somit ergibt sich ein systematischer Fehler für $n$ von $F_{sys} = \unit[0.017]{mmol}$, der Gesamtfehler ergibt sich demnach zu $F = \unit[0.022]{mmol}$. \\
Für die später benötigten Stoffmengen $n_1$ ($\unit[47.5]{^\circ C}$) und $n_3$ ($\unit[55]{^\circ C}$) ergeben sich folgende Werte.
\begin{align*}
n_1 &= \unit[(22.86 \pm 0.02)]{mmol}\\
n_3 &= \unit[(20.36 \pm 0.02)]{mmol}
\end{align*}

\section{Diskussion und Zusammenfassung}
Mithilfe von Laufzeitmessung wurde in den ersten beiden Versuchen die Schallgeschwindigkeit von Luft, Kupfer, PVC und Aluminium berechnet. Im ersten Versuch ergeben sich sehr große Fehlerwerte, innerhalb des Fehlerbereichs stimmen die Werte allerdings mit Literaturwerten überein, anders als im zweiten Versuch, in dem die Werte einigermaßen nah an den Literaturwerten liegen, allerdings außerhalb des -- recht kleinen -- Fehlerbereichs.\\
In Versuch drei ist die Schallgeschwindigkeit erstaunlich genau bestimmt worden vor allem unter der Berücksichtigung der Tatsache, dass Umgebungsgeräusche einen sehr großen Störfaktor gebildet haben (insbesondere bei hohen Frequenzen) und die Länge, für die Resonanzen entstehen, nach Augenmaß ermittelt wurde.


%\begin{appendix}
%\addcontentsline{toc}{section}{Anhang}
%\section{Fehlerrechnung}
\newcommand{\rg}{R_\mathrm{G}} % zu faul zum Schreiben
\newcommand{\stat}{\Delta} % ich ändere ca. jede Stunde meine Meinung wie ich statistische Fehler bezeichnen soll...

Den statistischen Fehler $\Delta \rg$ des Widerstands bei Schmelztemperatur (destilliertes Wasser) erhalten wir als die Standardabweichung des Mittelwerts aus den 3 Messungen:
%
\begin{align*}
            \rg &= \frac{6.66 + 6.66 + 6.67}{3} \\
        s_{\rg} &= \sqrt{\frac{\left(6.66 - \rg\right)^2 + \left(6.66 - \rg\right)^2 + \left(6.67 - \rg\right)^2}{3 - 1}} \\[1mm]
    \stat \rg  &= \frac{s_{\rg}}{\sqrt{3}} \approx \unit[0.0033]{k\Omega}
\end{align*}
%
Den statistischen Fehler $\stat R_\mathrm{G1}$ für die Salzlösung erhalten wir genauso und er ergibt sich ebenfalls zu $\Delta R_\mathrm{G1} \approx \unit[0.0033]{k\Omega}$.
%Sicherheitshalber berücksichtigen wir auch einen eventuellen systematischen Fehler im Termistor von $\pm \unit[0.005]{k\Omega}$ (halbe Skala).

Den Fehler für die Steigung der linearen Näherung an die Temperatur-Widerstand-Kurve des Termistors erhalten wir allein durch die lineare Regression.
%
Zusammen mit dem Wert $a = \unit[(-3.215 \pm 0.024)]{K\, k\Omega^{-1}}$ für die Steigung der linearen Näherung an die Temperatur berechnen wir die Verschiebung des Schmelzpunktes mithilfe von Gleichung~\ref{eq:DT}. Das Gaußsche Fehlerforpflanzungsgesetz liefert für die statistischen Fehler
%
\begin{align*}
    \stat(\Delta T_\mathrm{G}) &= \sqrt{(R_\mathrm{G} - R_\mathrm{G1})^2 \stat a^2 +
                                                      a^2 (\stat {R_\mathrm{G}}^2 + \stat {R_\mathrm{G1}}^2)} \\
                                              &= \unit[0.020]{K}
\end{align*}
%

Die Unsicherheit bei der Masse des Wassers ergibt sich durch
\[
    \stat(n M_1) = \sqrt{2} \cdot \unit[0.0005]{g} + 2 \cdot \unit[0.0002]{g} \approx \unit[0.0011]{g}
\]
Denn der statistische bzw. systematische Fehler der Waage betragen $\unit[0.0005]{g}$ bzw. $\unit[0.0002]{g}$ und das Gewicht des Wassers wurde indirekt durch wiegen des leeren Reagenzglases und später des vollen Reagenzglases ermittelt.

Der Fehler bei der Stoffmenge des Salzes $N$ ergibt sich aus dem statistischen plus dem systematischen Fehler der Waage, da die Unsicherheit der molaren Masse des Salzes mehrere Größenordnungen kleiner ist%
\footnote{Die molaren Massen von Na, N und O sind in Periodensystemen mit bis zu 4 Nachkommastellen tabelliert, d.\,h.\ der Fehler ist $<0.001\%$}%.

Aufgrund der hohen Genauigkeit der Waage sind diese Fehler im Folgenden jedoch irrelevant: 
%Vergleicht man den Beitrag von $\stat (\Delta T_\mathrm{G})$ und $\stat(n M_1)$ in Gleichung~\ref{eq:N'} sieht man:
%
%\begin{align*}
%    \frac{\partial N'}{\partial \Delta T_\mathrm{G}} \stat(\Delta T_\mathrm{G}) &= \frac{n M_1}{K_\mathrm{G1}}
%\end{align*}
%
In Gleichung~\ref{eq:N'} hat $\Delta T_\mathrm{G}$ einen relativen Fehler%
\footnote{Da es sich bei bei Gleichungen~\ref{eq:N'} und \ref{eq:alpha} grundsätzlich nur um Multiplikationen und Divisionen handelt kann man einfach die relativen Fehler vergleichen.}
von ca.\ $2\%$, $n M_1$ jedoch nur einen von ca.\ $0.005\%$. Folglich hat auch $N'$ in Gleichung~\ref{eq:alpha} einen relative Fehler von $2\%$ somit also deutlich mehr als $N$ mit nur ca.\ $0.1\%$. Die Fehler ergeben sich somit als
%
\begin{align*}
        \stat N' &= \frac{n M_1}{\unit[1]{kg}} \stat(\Delta K_\mathrm{G}) \approx \unit[0.22\cdot\! 10^{-3}]{mol} \\
    \stat \alpha &= \frac{\stat N'}{N (z-1)} \approx 0.016
\end{align*}
%







%\end{appendix}


\end{document}
