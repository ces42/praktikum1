\section{Bestimmung der Verdampfungsenthalpie}

Der Dampfdruck $p_d$ ist gerade der Druck der Plateaus in Abbildung~\ref{diag:pV}. Um $p_d$ zu erhalten wurden der Mittelwert aller Werte von $p$, von denen wir meinen, dass sie Teil des Plateaus sind, genommen. Um den Fehler $\Delta p_d$ abzuschätzen bilden wir die Standardabweichung des Mittelwerts dieser Werte und multiplizieren diese Pauschal mit einem Faktor von $4$ (Korrektur für Fehler beim "`Augenmaß"'). 


\begin{figure}    
    \begin{tikzpicture}
    \begin{axis}[
    height=\plotheight, width=\plotwidth,
    xlabel={$1/T$ in $\unit{K^{-1}}$}, ylabel={$p$ in $\unit[100]{kPa}$},
    ymode=log,
    only marks,
    % xmin=-10, xmax=800,
    legend entries={Messwerte, Regression}, legend pos=north east,
    ]
    \addplot+[error bars/.cd, y dir=both, x dir=both, y explicit, x explicit]
             table [x=t, y=p, x error=err_t, y error=err_p, col sep=comma] {./Daten/arrhenius.csv};
    \addplot+[domain=0.00316:0.00336, color=red, smooth, thick, mark size=0] {exp(10.75899086) * exp(-2279.861837*x)};
    \end{axis}
    
    \end{tikzpicture}
    
    \caption{$p$-$V$ Diagramm}
    \label{diag:arr}
\end{figure}

In Abbildung~\ref{diag:arr} sind die $5$ Dampfdrücke logarithmisch über $1/T$ aufgetragen. Die gewichtete Regression $p_d \propto e^{-m/T}$ ergibt einen Wert von $m = \unit[(2.28 \pm 0.07)]{K}$


