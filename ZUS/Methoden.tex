\subsection{Verwendete Methoden}
\subsubsection{ideale Gase}
Unter idealen Gasen versteht man Gase, dessen Moleküle nicht miteinander wechselwirken und die kein Eigenvolumen besitzen. Solche Gase existieren in der Realität nicht, für Gase unter sehr geringem Druck kann dieses Modell allerdings als gute Näherung verwendet werden. Für ideale Gase gilt folgende Gleichung:
\begin{equation}
p \cdot V = n \cdot R \cdot T
\end{equation}
wobei $p$ der Druck, $V$ das Volumen $n$ die Stoffmenge, $R$ die allgemeine Gaskonstante $R=8,3144621 \frac{\text{\fontsize{9}{1}\selectfont {J}}}{\text{\fontsize{9}{1}\selectfont {mol K}}}$