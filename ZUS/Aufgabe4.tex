\section{Aufgabe 4}

\newcommand{\plateau}[3]{
\draw[black, thick](axis cs:#1,#3)--(axis cs:#2,#3);
\draw[black, thick](axis cs:#1,#3 + 0.4) --(axis cs:#1,#3 - 0.4);
\draw[black, thick](axis cs:#2,#3 + 0.4) --(axis cs:#2,#3 - 0.4);
}


Wir berechnen die molaren Volumina durch $V_\mathrm{m} = V / n$.

\begin{figure}    
    \begin{tikzpicture}
    \begin{axis}[
    height=\plotheight, width=\plotwidth,
    xlabel={$V$ in $\unit{dm^3\, mol^{-1}}$}, ylabel={$p$ in $\unit[100]{kPa}$},
    %caled x ticks,
    % only marks,
    %xmin=-10, xmax=800,
    legend entries={$T = \unit[25]{ ^\circ C}$, $T = \unit[30]{ ^\circ C}$, $T = \unit[35]{ ^\circ C}$,
                    $T = \unit[40]{ ^\circ C}$, $T = \unit[42.5]{ ^\circ C}$, $T = \unit[45]{ ^\circ C}$,
                    $T = \unit[47.5]{ ^\circ C}$, $T = \unit[50]{ ^\circ C}$, $T = \unit[55]{ ^\circ C}$,
                    kritischer Punkt}, legend pos=north east,
    ]
    \addplot+[mark size=1.] table [x=V_m, y=p, col sep=comma] {./Daten/molar25.csv};
    \addplot+[mark size=1.] table [x=V_m, y=p, col sep=comma] {./Daten/molar30.csv};
    \addplot+[mark size=1.] table [x=V_m, y=p, col sep=comma] {./Daten/molar35.csv};
    \addplot+[mark size=1.] table [x=V_m, y=p, col sep=comma] {./Daten/molar40.csv};
    \addplot+[mark size=1.] table [x=V_m, y=p, col sep=comma] {./Daten/molar42.5.csv};
    \addplot+[mark size=1.] table [x=V_m, y=p, col sep=comma] {./Daten/molar45.csv};
    \addplot+[mark size=1.] table [x=V_m, y=p, col sep=comma] {./Daten/molar47.5.csv};
    \addplot+[mark size=1.] table [x=V_m, y=p, col sep=comma] {./Daten/molar50.csv};
    \addplot+[mark size=1.] table [x=V_m, y=p, col sep=comma] {./Daten/molar55.csv};
    
    
    %\draw[black, thick](axis cs:0.133,25.5)--(axis cs:0.569,25.5);
    %\draw[black, thick](axis cs:0.133,25.9) --(axis cs:0.133,25.1);
    %\draw[black, thick](axis cs:0.569,25.9) --(axis cs:0.569,25.1);
    
    % 25 Grad
    \plateau{0.142}{0.665}{22.83}
    % 30 Grad
    \plateau{0.142}{0.603}{25.28}
    % 35 Grad
    \plateau{0.179}{0.498}{29.29}
    % 40 Grad
    \plateau{0.160}{0.408}{32.43}
    % 42.5 Grad
    \plateau{0.178}{0.355}{34.40}
    \addplot[ color=green!90!black, only marks, mark size=2, error bars/.cd, y dir=both, x dir=both, y explicit, x explicit,]  coordinates{
        (0.245,36.7) +- (0.04, 1.2)
    };	
    
    
   \end{axis}
   
    \end{tikzpicture}
    
    \caption{$p$-$V$ Diagramm}
    \label{diag:pV}
\end{figure}


Abbildung~\ref{diag:pV} Zeigt alle aufgenommenen Isobaren. 

Wir betrachten nur den Fehler der Messpunkte unserer Gruppe.
Den Gesamtfehler der Volumenmessung schätzen wir auf $\Delta V \approx \unit[0.025]{cm^3}$ (halbe Skala). Der relative Fehler liegt somit in der Größenordnung von $1\%$ und ist somit viel größer als der Fehler in der Stoffmenge. Wir können demnach für den Fehler des molaren Volumens $\Delta V_m \approx \Delta V / n$ ansetzen.

Den statistischen Fehler der Druckmessung schätzen wir auf $\unit[0.25 \e{5}]{Pa}$, Fehler des Messgeräts beträgt $\unit[0.5 \e5]{Pa}$ ($1\%$ des maximalen Ausschlags).

Es wurde versucht grob zu erkennen, wo bei den subkritischen Temperaturen die Plateaus liegen. Dies sind die Bereiche, in denen sowohl Gaß als auch Flüssigkeit vorliegen. Bei der $\unit[45]{ ^\circ C}$-Isotherme ist nicht mehr klar ob ein Plateau vorliegt. Wir schätzen den kritischen Punkt und seinen Fehler auf
\begin{align*}
    p_\mathrm{crit} &= \unit[(3.67 \pm 0.12)]{MPa} \\
    V_\mathrm{crit} &= \unit[(0.245 \pm 0.04)]{dm^3\, mol^{-1}}
\end{align*}


