\section{Diskussion und Zusammenfassung}
In diesem Versuch wurde vor allem der theoretische und tatsächliche Verlauf von Isothermen bei Volumen- und Temperaturänderung behandelt. Wichtige Schlüsse aus diesem Versuch sind die Abweichung von der Theoriekurve (nach Van der Waals), durch die Tatsache, dass sich das Gas unter hohem Druck unterhalb der kritischen Temperatur teilweise verflüssigt und damit den Druck für einen begrenzten Bereich so ausgleicht (genaueres wurde bereits in den verwendeten Methoden behandelt), dass dieser in diesem konstant bleibt. Die Existenz einer solchen kritischen Temperatur und eines dazugehörigen kritischen Drucks (erste und zweite Ableitung sind an diesem Punkt null) ist ebenfalls eine wichtige Erkenntnis.\\
Die Fehlerwerte für die Stoffmenge (siehe Sektion \ref{sec:Stoffmenge}) sind ziemlich klein, was durch die hohe Anzahl von Werten herrührt. Allerdings können diese abweichen, da der Fehlerwert auch von der Art der Extrapolation abhängt (welche nicht vorgegeben ist).\\
Dierecht großen Fehler für die Bestimmung der kritischen Temperatur von $SF_6$ und der Verdampfungsenthalpie ergeben sich durch die nur durch Augenmaß bestimmbaren ''Plateaus'' (Koexistenzbereich von Flüssigkeit und Gas) und die ebenfalls nur durch Augenmaß bestimmbare kritische Temperatur durch den Verlauf des Gas- bzw. Flüssigkeitsvolumens. Selbiges gilt für den kritischen Druck. Zudem sind die durch den menschlichen Faktor erzeugten Fehler (insbesondere die anderer Gruppen) nur schwer abzuschätzen und werden daher eher größer geschätzt. Ein weiterer Faktor sind teilweise abweichende Messungen von der Erwartung, die sich vermutlich unter anderem durch eine schnelle Messung bedingen (insbesondere bei der Messung für ein sich vergrößerndes Volumen), da die Flüssigkeit noch nicht vollständig verdampfen konnte, was den Druck beeinträchtigt.