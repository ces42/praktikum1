\section{Verwendete Methoden}
\subsection{Lösungen}
Unter Dissoziation versteht man den Zerfall eines Moleküls in mindestens zwei geladene oder neutrale Bruchstücke. Diese Bruchstücke wechselwirken nun mit den Molekülen des Lösungsmittels (Lösung). Bei einer idealen Lösung überwiegt der Anteil des Lösungsmittels deutlich gegenüber dem zu lösenden Stoff. Allerdings werden meist nicht alle Moleküle des zu lösenden Stoffes aufgespalten. Der Anteil der aufgespaltenen Moleküle zur Gesamtmenge an Molekülen nennt man den Dissoziationsgrad $\alpha$. Zerfällt ein Molekül in z Moleküle und beschreibt N die Anzahl der Moleküle des Lösungsstoffes sowie N' die Anzahl der gelösten Moleküle, so ergibt sich folgender Zusammenhang:

\begin{equation}
N' = N \cdot \alpha \cdot z + N(1-\alpha) \label{eq:N'}
\end{equation}
und somit
\begin{equation}
\alpha = \frac{\frac{N'}{N}-1}{z-1} \label{eq:alpha}
\end{equation}
%
Mithilfe der Anzahl der gelösten Teilchen, kann also der Dissoziationsgrad ermittelt werden. Hierbei wird von der  François Marie Raoult gefundenen Gesetzmäßigkeit, welche den Dampfdruck mit der Teilchenanzahl verknüpft, Gebrauch gemacht. Mithilfe dieser Gesetzmäßigkeit und der Clausius-Clapeyronschen Beziehung, ergibt sich unter der in der Aufgabenstellung gegebenen Näherung, dass Schmelzdruck-Kurven des Lösungsmittels und der Lösung nahezu parallel sind, und der Vereinfachung, dass die Lösung stark verdünnt ist, folgender Zusammenhang: \footnote{Siehe Aufgabenstellung}

\begin{equation}
\frac{N'}{n} = \frac{\Delta T_G \cdot M_1}{K_{G1} \cdot \unit[1]{kg}}
\end{equation}

