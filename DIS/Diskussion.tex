\section{Diskussion und Zusammenfassung}
In diesem Experiment wurde der Gefrierpunkt von destilliertem Wasser und einer Salzlösung (mit $NaNO_3$) gemessen. Festgestellt wurde außerdem, dass die Testflüssigkeit ihren Gefrierpunkt kurzzeitig überschreitet, sobald der Gefriervorgang einsetzt allerdings sehr schnell auf ihren Gefrierpunkt steigt und während des Gefriervorgangs dann konstant bleibt. Dies liegt an der Tatsache, dass bei dem Gefriervorgang wiederum Energie freigesetzt wird, allerdings steigt die Temperatur dadurch nicht über den Gefrierpunkt, wodurch sich auch die kurzzeitige Unterschreitung des Gefrierpunktes erklären lässt.\\
Für den Gefrierpunkt von Wasser wurde $\approx 0.1^\circ $C ermittelt. Mit dem bekannten Gefrierpunkt von $\unit[0]{^\circ C}$ und der berechneten Ungenauigkeit von $\approx \pm \unit[0.19]{K}$ \footnote{Setzt sich zusammen aus der im Anhang berechneten Schwankung der Temperaturdifferenz plus dem Fehler von $b$, welcher aus der Thermistorkalibrierung stammt (Die Korrelation dieser beiden Ungenauigkeiten ist in diesem Fall vernachlässigbar klein)} liegt das richtige Ergebnis innerhalb der Ungenauigkeit und ist daher sinnvoll. Auch der Gefrierpunkt der Lösung von $\approx -1.0^\circ$C ist wie erwartet etwas unter dem Gefrierpunkt von Wasser und damit ebenfalls realistisch.







