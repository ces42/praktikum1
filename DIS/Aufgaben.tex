\section{Fragen}
Was für Wasser gesagt wird gilt natürlich für alle polare Lösungsmittel.

\subsection{Elektrolyte}
Elektrolyte dissoziieren in geladene Ionen. Für eine ideale Lösung sollten die Partikel des gelösten Stoffes nicht miteinander wechselwirken (sondern nur mit dem Lösungsmittel). Die Ionen interagieren jedoch durch ihre Coulomb-Potentiale ($r^{-1}$-Abhängigkeit) miteinander. Dieses Potential kann auch die Hydrathüllen der Ionen nicht allzu gut ausgeglichen werden, da die Dipolpotentiale der Wassermoleküle eine $r^{-2}$ Abhängigkeit haben und somit schneller abfallen als das Potential der Ionen.

\subsection{Dampfdruck}
Dies gilt zumindest für elektrolytische Lösungen. Die Anziehung zwischen den Ionen und den Wassermolekülen führt dazu das die Wassermoleküle tendenziell weniger aus der Flüssigkeit verdampfen. Man kann auch sagen, dass die Anziehungskräfte zwischen den Molekülen in der Lösung größer sind als in destilliertem Wasser (weil die Ionen eine starke Anziehung ausüben) und deswegen weniger verdampft.

\subsection{Kältebad}
Wie man in diesem Versuch gelernt hat senken Salze den Schmelzpunkt von Wasser. Im Kältebad kann genug Salz den Schmelzpunkt unter die Temperatur des Eises senken, weshalb das Eis schmilzt. Dabei entzieht es seiner Umgebung, also dem Kältebad, die Schmelzwärme. Dies ist der Haupteffekt, der zur Abkühlung führt.\\
Ein anderer Effekt ist noch, dass die Energiebilanz der Dissoziation des Viehsalzes leicht endotherm ist, d.\,h.\ sie kühlt das Wasser auch minimal ab.\\
Man kann grundsätzlich jedes Salz verwenden, solange es stark genug dissoziiert um den Schmelzpunkt merklich zu verringern und nicht sehr viel Energie bei der Dissoziation freisetzt.

\subsection{Elektrischer Widerstand}
Elektrische Halbleiter sind Stoffe, die sowohl Eigenschaften von Isolatoren, also auch von Leitern besitzen. Im Allgemeinen steigt die elektrische Leitfähigkeit bei steigender Temperatur. Dies liegt am chemischen Aufbau von Halbleitern. Die benachbarten Atome in den Halbleitern werden durch die Atombindung zu einem Gitter zusammengefügt. Bei steigender Temperatur können Elektronen aus dieser Bindung ausbrechen und sich somit frei im Gitter bewegen, was zu einer steigenden Elektrischen Leitfähigkeit führt und somit zu einer großen Veränderung des elektrischen Widerstands.\\
Bei Metallen hingegen liegen die Elektronen in einem Elektronengas vor, das bedeutet dass diese sehr gute Leiter sind. Bei Metallen sinkt die Leitfähigkeit mit steigender Temperatur nur durch den Umstand, dass die immer stärker schwingenden Atome die Bewegungsgeschwindigkeit der Elektronen verringern. Diese Eigenschaft wirkt allerdings deutlich schwächer, da sich die Struktur durch eine Temperaturänderung nicht verändert, wie es bei Halbleitern der Fall ist.









