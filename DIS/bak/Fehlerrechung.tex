\section{Fehlerrechnung}
\newcommand{\rg}{R_\mathrm{G}} % zu faul zum Schreiben
\newcommand{\stat}{\Delta} % ich ändere ca. jede Stunde meine Meinung wie ich statistische Fehler bezeichnen soll...

Den statistischen Fehler $\Delta \rg$ des Widerstands bei Schmelztemperatur (destilliertes Wasser) erhalten wir als die Standardabweichung des Mittelwerts aus den 3 Messungen:
%
\begin{align*}
            \rg &= \frac{6.66 + 6.66 + 6.67}{3} \\
        s_{\rg} &= \sqrt{\frac{\left(6.66 - \rg\right)^2 + \left(6.66 - \rg\right)^2 + \left(6.67 - \rg\right)^2}{3 - 1}} \\[1mm]
    \stat \rg  &= \frac{s_{\rg}}{\sqrt{3}} \approx \unit[0.0033]{k\Omega}
\end{align*}
%
Den statistischen Fehler $\stat R_\mathrm{G1}$ für die Salzlösung erhalten wir genauso und er ergibt sich ebenfalls zu $\Delta R_\mathrm{G1} \approx \unit[0.0033]{k\Omega}$.
%Sicherheitshalber berücksichtigen wir auch einen eventuellen systematischen Fehler im Termistor von $\pm \unit[0.005]{k\Omega}$ (halbe Skala).

Den Fehler für die Steigung der linearen Näherung an die Temperatur-Widerstand-Kurve des Termistors erhalten wir allein durch die lineare Regression.
%
Zusammen mit dem Wert $a = \unit[(-3.215 \pm 0.024)]{K\, k\Omega^{-1}}$ für die Steigung der linearen Näherung an die Temperatur berechnen wir die Verschiebung des Schmelzpunktes mithilfe von Gleichung~\ref{eq:DT}. Das Gaußsche Fehlerforpflanzungsgesetz liefert für die statistischen Fehler
%
\begin{align*}
    \stat(\Delta T_\mathrm{G}) &= \sqrt{(R_\mathrm{G} - R_\mathrm{G1})^2 \stat a^2 +
                                                      a^2 (\stat {R_\mathrm{G}}^2 + \stat {R_\mathrm{G1}}^2)} \\
                                              &= \unit[0.020]{K}
\end{align*}
%

Die Unsicherheit bei der Masse des Wassers ergibt sich durch
\[
    \stat(n M_1) = \sqrt{2} \cdot \unit[0.0005]{g} + 2 \cdot \unit[0.0002]{g} \approx \unit[0.0011]{g}
\]
Denn der statistische bzw. systematische Fehler der Waage betragen $\unit[0.0005]{g}$ bzw. $\unit[0.0002]{g}$ und das Gewicht des Wassers wurde indirekt durch wiegen des leeren Reagenzglases und später des vollen Reagenzglases ermittelt.

Der Fehler bei der Stoffmenge des Salzes $N$ ergibt sich aus dem statistischen plus dem systematischen Fehler der Waage, da die Unsicherheit der molaren Masse des Salzes mehrere Größenordnungen kleiner ist%
\footnote{Die molaren Massen von Na, N und O sind in Periodensystemen mit bis zu 4 Nachkommastellen tabelliert, d.\,h.\ der Fehler ist $<0.001\%$}%.

Aufgrund der hohen Genauigkeit der Waage sind diese Fehler im Folgenden jedoch irrelevant: 
%Vergleicht man den Beitrag von $\stat (\Delta T_\mathrm{G})$ und $\stat(n M_1)$ in Gleichung~\ref{eq:N'} sieht man:
%
%\begin{align*}
%    \frac{\partial N'}{\partial \Delta T_\mathrm{G}} \stat(\Delta T_\mathrm{G}) &= \frac{n M_1}{K_\mathrm{G1}}
%\end{align*}
%
In Gleichung~\ref{eq:N'} hat $\Delta T_\mathrm{G}$ einen relativen Fehler%
\footnote{Da es sich bei bei Gleichungen~\ref{eq:N'} und \ref{eq:alpha} grundsätzlich nur um Multiplikationen und Divisionen handelt kann man einfach die relativen Fehler vergleichen.}
von ca.\ $2\%$, $n M_1$ jedoch nur einen von ca.\ $0.005\%$. Folglich hat auch $N'$ in Gleichung~\ref{eq:alpha} einen relative Fehler von $2\%$ somit also deutlich mehr als $N$ mit nur ca.\ $0.1\%$. Die Fehler ergeben sich somit als
%
\begin{align*}
        \stat N' &= \frac{n M_1}{\unit[1]{kg}} \stat(\Delta K_\mathrm{G}) \approx \unit[0.22\cdot\! 10^{-3}]{mol} \\
    \stat \alpha &= \frac{\stat N'}{N (z-1)} \approx 0.016
\end{align*}
%






