\subsection{Berechnung des Trägheitsmoments durch Approximation des Körpers}

\begin{figure}
\begin{minipage}{.49\textwidth}      
    \centering
    \includegraphics[scale=.28]{./Bilder/figur1.jpeg}
    \caption{Figur 1}
    \label{fig:1}
\end{minipage}
\begin{minipage}{.49\textwidth}
    \centering
    \includegraphics[scale=.28]{./Bilder/figur2.jpeg}
    \caption{Figur 2}
    \label{fig:2}
\end{minipage}
\end{figure}


Für Person 1 und Figur 1 errechnen%
\footnote{für eine Ausführliche Rechnung siehe \url{https://github.com/ces42/praktikum1/blob/master/TRM/Daten/Auswertung.ipynb}}
wir ein theoretisches Trägheitsmoment von  $\unit[0.8]{kg\,m^2}$. Für Person 2 und Figur 2 erhalten wir einen Wert von $\unit[1.9]{kg\,m^2}$. 

\subsection{Berechnung des Trägheitsmoments durch Extrapolation der Puppe}



