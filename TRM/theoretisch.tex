\subsection{Berechnung des Trägheitsmoments durch Approximation des Körpers}

\begin{figure}
\begin{minipage}{.49\textwidth}
    \centering
    \includegraphics[scale=.28]{./Bilder/figur1.jpeg}
    \caption{Figur 1}
    \label{fig:1}
\end{minipage}
\begin{minipage}{.49\textwidth}
    \centering
    \includegraphics[scale=.28]{./Bilder/figur2.jpeg}
    \caption{Figur 2}
    \label{fig:2}
\end{minipage}
\end{figure}

Für Person 1 und Figur 1 errechnen%
\footnote{für eine Ausführliche Rechnung siehe \url{https://github.com/ces42/praktikum1/blob/master/TRM/Daten/Auswertung.ipynb}}
wir ein theoretisches Trägheitsmoment von  $\unit[0.8]{kg\,m^2}$. Für Person 2 und Figur 2 erhalten wir einen Wert von $\unit[1.9]{kg\,m^2}$. 


\begin{table}
\begin{center}
\begin{tabular}{|c|c|c|c|c|c|c|}
\hline 
• & Masse & Höhe & Bein Länge & Bein Radius & Arm Länge & Arm Radius \\
\hline 
Person 1 & 61.4 kg & 185 cm & 94 cm & 6.8 cm & 69 cm & 4.1 cm \\
\hline 
Person 2 & 74.6 kg & 184 cm & 49/60 cm & 6.9 cm & 37/39 cm & 4.9 cm \\
\hline
\end{tabular}
\end{center}

\begin{center}
\begin{tabular}{|c|c|c|c|c|}
\hline
• & Hüfte Höhe & Hüfte Breite & Hüfte Dicke & Kopf Radius \\
\hline
Person 1 & 63 cm & 35 cm & 22 cm & 9.4 cm \\
\hline
Person 2 & 56 cm & 32 cm & 22 cm & 9.4 cm \\ 
\hline
\end{tabular}
\caption{Gemessene Werte für die Annäherung (bei zwei Werten mit ''/'' getrennt ist jeweils Ober-/Unterarm bzw. Ober-/Unterschenkel gemeint)}
\end{center}

\end{table}

\subsection{Berechnung des Trägheitsmoments durch Extrapolation der Puppe}
Angenommen wird eine homogene Dichteverteilung, beim Menschen wird diese in etwa $\rho_M \approx (1.05 \pm 0.5) \frac{g}{cm^3}$ betragen (leicht größere Dichte als Wasser) und bei der Puppe etwa $\rho_P \approx (0.75 \pm 0.1) \frac{g}{cm^3}$ (Dichte von Holz).\\
Nehmen wir an, dass die Puppe in etwa die selben Proportionen wie die Menschen haben, kann das Trägheitsmoment mithilfe von Formel \ref{eq:int} extrapoliert werden. Wir erhalten einen Wert von: $J \approx \unit[1.7]{kg \, m^2}$ für die zweite Figur. Dieser Wert liegt relativ nahe an dem experimentell bestimmten Wert. Zudem gibt es einen sehr großen Fehler durch die vielen Abschätzungen. Für die erste Figur liegt der Wert allerdings mit $J \approx \unit[0.3]{kg \, m^2}$ ziemlich weit daneben. Zusammen mit der unrealistisch großen Änderung des Trägheitsmomentes von Figur eins zu Figur zwei (siehe Sektion \ref{sec:puppe}) liegt der Schluss nahe, dass dieser Wert für die Puppe nicht richtig ist.


