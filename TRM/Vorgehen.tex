\section{Experimentelles Vorgehen}
Zur Bestimmung der Trägheitsmomente wird Formel \ref{eq:traeg} benutzt. Mithilfe dieser Formel kann das Trägheitsmoment durch die Schwingungsdauer $T$ berechnet werden, welche zwei Scheiben (eine für die Puppe, eine für den Menschen) bei einer Auslenkung aus der Ruhelage haben.\\
Hierfür muss allerdings zu Beginn die Winkelrichtgröße $d$ (kleine Scheibe) bzw. $D$ (große Scheibe) ermittelt werden. Hierfür werden zwei Verfahren angewendet. Zum einen die statische Methode, bei der für verschiedene Winkel die Kraft, welche (senkrecht zur Drehachse) für einen (festen) Radius $r$ wirkt. Die zweite Methode ist die Dynamische. Hierfür wird die Schwingungsdauer der leeren Scheibe, sowie die der Scheibe mit zusätzlichen Gewichten gemessen. Die Änderung des Trägheitsmomentes ($J_Z$) durch die Gewichte kann rechnerisch ermittelt werden. Durch den Zusammenhang
\begin{equation}
T^2 = \frac{4\pi^2}{d} \cdot (J_0 + J_Z)
\end{equation}
kann durch die Steigung einer Ausgleichsgerade von $T^2$ in Abhängigkeit von dem Trägheitsmoment ermittelt werden, da die Steigung dieser Gerade genau $\frac{4\pi^2}{d}$ entspricht. Durch die selbe Formel kann durch Extrapolation auf den y-Achsenabschnitt ($J_Z = 0$)  $J_0$ ermittelt werden.\\
Nach der Kalibrierung kann der Trägheitsmoment allein durch die Messung der Schwingungsdauer berechnet werden. In diesem Versuch soll dieses für (mindestens) zwei verschiedene Figuren für die Puppe und den Menschen bestimmt werden (die gleichen für Puppe und Mensch). Anschließend soll der Trägheitsmoment der Puppe für die gleiche Figur auf das des Menschen extrapoliert werden (siehe Formel \ref{eq:int}).





