\subsection{Trägheitsmoment der Puppe}

%TODO richtige Werte
% d = (22.33 +- 0.15) mN m
% d = (23.9 +- 0.10) mN m

Für die Perioden der Schwingung für die Figuren 1 bzw. 2 (vgl. Abb.~\ref{fig:1} und \ref{fig:2}) bei der Puppe erhalten wir 
%
\begin{align*}
    T_1 &= \unit[(769.6 \pm 1.2)]{ms} \\
    T_2 &= \unit[(892.6 \pm 1.2)]{ms}
\end{align*}
%
wobei sich die statistischen Fehler aus der Standardabweichung des Mittelwerts ergeben. Der Systematische Fehler der Lichtschranke sollte bei maximal $\unit[1]{ms}$ (1 digit) liegen. Das Trägheitsmoment berechnet sich durch
\[
    J = \frac{d \cdot T^2}{4 \pi^2} 
\]
womit wir erhalten
%
\begin{align*}
    J_1 &= \unit[(3.585 \pm 0.025) \e{-4}]{kg\,m^2} \\
    J_2 &= \unit[(4.823 \pm 0.031) \e{-4}]{kg\,m^2}
\end{align*}
%
Bei der zweiten Figur ist das Trägheitsmoment somit ca $35 \%$ größer.


\subsection{Trägheitsmoment der Menschen}

Für die Perioden der Schwingungen erhalten wir $T_1 = \unit[(5.49 \pm 0.06)]{s}$ für Figur 1 und $T_2 = \unit[(7.00 \pm 0.13)]{s}$. Damit ergeben sich, unter Berücksichtigung des Eingenträgheitsmoments der Scheibe%
\footnote{Als Zylinder genähert}
von $\unit[(0.6843 \pm 0.0023)]{kg\,m^2}$ Trägheitsmomente von 
%
\begin{align*}
    J_1 &= \unit[(0.97 \pm 0.05)]{kg\, m^2}\\
    J_2 &= \unit[(2,.00 \pm 0.11)]{kg\, m^2}
\end{align*}
%
Die Fehler ergeben sich mit Gauß'scher Fortpflanzung aus dem Fehler der Zeitmessung und dem Fehler der Winkelrichtgröße. Der Fehler des Eigenträgheitsmoments hat kaum einen Einfluss.

