\section{Aufgaben}
\paragraph{Energieformen}
Bei einer Schaukel existieren kinetische sowie potentielle (Höhe) Energie.
\paragraph{Drehimpuls}
Der Drehimpuls ist minimal, wenn sich die Schaukel nicht bewegt, also an ihrem höchsten Punkt.
\paragraph{Beeinflussung des Trägheitmomentes}
Durch die Erniedrigung des Schwerpunktes wird der Radius vergrößert und demzufolge das Trägheitsmoment erhöht.
\paragraph{Veränderung der Energie}
Bei der Absenkung des Oberkörpers wird zum einen die Höhenenergie der Höhendifferenz in kinetische Energie umgewandelt, zusätzlich wird durch die Bewegung weitere kinetische Energie zugefügt.
\paragraph{Energiezufuhr zum System}
Wenn der Schaukelnde an einem Höhepunkt der Schaukel seinen Schwerpunkt erniedrigt, wird die entsprechende Höhenenergie in kinetische Energie umgewandelt, also ist der Höhepunkt am zweiten Maxima der Schaukel wieder auf gleicher Höhe wie der ursprüngliche Höhepunkt. Durch (schnelles) Aufrichten, also Erhöhung des Schwerpunktes kann nun Energie in Form von Höhenenergie zugefügt werden. Dies kann periodisch wiederholt werden.
\paragraph{Verhalten für Drehimpuls/maximalen Drehwinkel} Da dem System kontinuierlich Energie hinzugefügt wird, erhöht sich dementsprechend der maximale Drehimpuls und also auch der maximale Drehwinkel (höhere Höhe kann durch mehr Energie erreicht werden).